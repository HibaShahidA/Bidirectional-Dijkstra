\documentclass[10pt]{article}
\usepackage[utf8]{inputenc}
\usepackage{graphicx}
\usepackage{hyperref}
\usepackage{booktabs}
\usepackage{array}
\usepackage[a4paper, margin=0.75in]{geometry}

\begin{document}

\begin{center}
    {\LARGE \textbf{Project Proposal: Bidirectional Dijkstra's Algorithm}} \\
    \vspace{0.5em}
    {\large \textbf{Team:} Qurba Mushtaq (08232), Hiba Shahid (08036)} \\
    {\large \textbf{Date:} \today}
\end{center}

\section*{Paper Details}
\textbf{Title:} Bidirectional Dijkstra's Algorithm is Instance-Optimal\\
\textbf{Authors:} Bernhard Haeupler et al.\\
\textbf{Conference:} SOSA 2025\\
\textbf{DOI:} \url{https://epubs.siam.org/doi/10.1137/1.9781611978315.16}

\section*{Summary}
This paper proves the \textbf{instance-optimality} of bidirectional Dijkstra's algorithm for shortest-path computations, showing that no correct algorithm outperforms it by more than a constant factor.

\textbf{Key contributions:}
\begin{itemize}
    \item Formal proof of instance-optimality.
    \item Near-optimal guarantees for unweighted graphs.
    \item Comparative analysis with A* search.
\end{itemize}

\section*{Justification and Feasibility}
\textbf{Relevance:} Establishes performance bounds for graph algorithms; applicable in routing systems.\\
\textbf{Implementation:} Pseudocode provided, enabling direct translation into code.\\
\textbf{Resources:} No reference implementation; datasets available (Kaggle, OpenStreetMap).

\section*{Implementation Plan}
\textbf{Algorithm:} Implements bidirectional search with adjacency lists and priority queues.\\
\textbf{Verification:} Compare with standard Dijkstra and A*; validate with real-world graphs.\\
\textbf{Challenges and Mitigation:}\\
\begin{tabular}{p{6cm}|p{6cm}}
\toprule
\textbf{Challenge} & \textbf{Mitigation} \\
\midrule
Termination conditions & Step-by-step validation \\
Bidirectional sync & Thread-safe structures \\
Large graphs & Progressive testing \\
\bottomrule
\end{tabular}

\section*{Team Responsibilities}
\textbf{Qurba Mushtaq:} Algorithm implementation, benchmarking, paper analysis.\\
\textbf{Hiba Shahid:} Graph generation, result analysis, report writing.

\section*{GitHub Repository}
\textbf{URL:} \url{https://github.com/HibaShahidA/Bidirectional-Dijkstra}\\
\textbf{Structure:} \texttt{/src} (code), \texttt{/data} (datasets), \texttt{/benchmarks} (performance scripts), \texttt{/docs} (notes).

\section*{Next Steps}
\begin{enumerate}
    \item Implement Algorithm 2.
    \item Develop graph generators.
    \item Compare with Dijkstra and A*.
    \item Analyze and document results.
\end{enumerate}

\end{document}
