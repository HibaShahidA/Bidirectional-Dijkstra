\documentclass[12pt]{article}
\usepackage[utf8]{inputenc}
\usepackage{graphicx}
\usepackage{hyperref}
\usepackage{booktabs}
\usepackage{array}

\title{Project Proposal: Implementation of Bidirectional Dijkstra's Algorithm}
\author{Team Members: Qurba Mushtaq 08232, Hiba Shahid 08036}
\date{\today}

\begin{document}

\maketitle

\section*{Paper Details}
\begin{itemize}
    \item \textbf{Title:} Bidirectional Dijkstra's Algorithm is Instance-Optimal
    \item \textbf{Authors:} Bernhard Haeupler, Richard Hladik, Vaclav Rozhon, Robert E. Tarjan, Jakub Tetek
    \item \textbf{Conference:} Proceedings of SOSA (Symposium on Simplicity in Algorithms)
    \item \textbf{Year:} 2025
    \item \textbf{DOI/Link:} \url{https://epubs.siam.org/doi/10.1137/1.9781611978315.16}
\end{itemize}

\section{Summary}
This paper provides a theoretical foundation for the efficiency of \textbf{bidirectional Dijkstra's algorithm}, proving its \textbf{instance-optimality} for shortest-path computations in both weighted and unweighted graphs. The authors demonstrate that in the adjacency list query model, no correct algorithm can outperform their implementation by more than a constant factor on any input graph.

Key contributions include:
\begin{itemize}
    \item Formal proof of instance-optimality in weighted graphs
    \item Near-optimal guarantees for unweighted graphs (within factor $O(\Delta)$)
    \item Comparative analysis with A* search
\end{itemize}

\section{Justification}
\subsection{Theoretical Significance}
\begin{itemize}
    \item Establishes rigorous performance bounds for fundamental algorithmic technique
    \item Bridges theory with practical applications in routing systems
\end{itemize}

\subsection{Pedagogical Value}
\begin{itemize}
    \item Reinforces core graph algorithm concepts
    \item Explores advanced topics like instance optimality
\end{itemize}

\subsection{Implementation Potential}
\begin{itemize}
    \item Clear pseudocode (Algorithm 2) provided
    \item Natural comparison points against standard algorithms
\end{itemize}

\section{Implementation Feasibility}
\subsection{Algorithm Specification}
\begin{itemize}
    \item Complete pseudocode with:
    \begin{itemize}
        \item Bidirectional search mechanics
        \item Termination conditions
        \item Path reconstruction logic
    \end{itemize}
\end{itemize}

\subsection{Implementation Complexity}
\begin{itemize}
    \item Standard graph structures (adjacency lists)
    \item Basic components (priority queues, distance tracking)
    \item No exotic dependencies (Python/Java/C++ compatible)
\end{itemize}

\subsection{Verification Methodology}
\begin{itemize}
    \item Comparison with standard Dijkstra
    \item Path verification in real-world graphs
    \item Stress testing with edge cases
\end{itemize}

\subsection{Resource Availability}
\begin{itemize}
    \item \textbf{Code:} Pseudocode available (no reference implementation)
    \item \textbf{Data:} Real-world graphs (Kaggle) + generated graphs
\end{itemize}

\subsection{Risk Mitigation}
\begin{center}
\begin{tabular}{p{6cm}|p{6cm}}
\toprule
\textbf{Challenge} & \textbf{Mitigation Strategy} \\
\midrule
Termination condition complexity & Step-by-step validation \\
Bidirectional synchronization & Thread-safe structures \\
Large graph handling & Progressive testing \\
\bottomrule
\end{tabular}
\end{center}

\section{Team Responsibilities}
\begin{center}
\begin{tabular}{p{0.48\textwidth}|p{0.48\textwidth}}
\toprule
\textbf{Qurba Mushtaq} & \textbf{Hiba Shahid} \\
\midrule
Core algorithm implementation & Graph generation and dataset curation \\
Performance benchmarking & Results analysis and visualization \\
Paper analysis & Report writing \\
\bottomrule
\end{tabular}
\end{center}

\section{GitHub Repository}
\begin{itemize}
    \item \textbf{URL:} \url{https://github.com/HibaShahidA/Bidirectional-Dijkstra}
    \item \textbf{Structure:}
    \begin{itemize}
        \item \texttt{/src} - Implementation code
        \item \texttt{/data} - Graph datasets
        \item \texttt{/benchmarks} - Performance scripts
        \item \texttt{/docs} - Technical notes
    \end{itemize}
\end{itemize}

\section{Next Steps}
\begin{enumerate}
    \item Implement Algorithm 2 with termination conditions
    \item Develop graph generators
    \item Design comparison experiments:
    \begin{itemize}
        \item Unidirectional Dijkstra
        \item A* search
    \end{itemize}
    \item Analyze results
\end{enumerate}

\end{document}
